\section{Zusammenfassung}

Durch den entwickelten Prototyp, welcher dem Nutzer Datenschutz und Kontrollmöglichkeiten bietet, konnten einige Rückschlüsse gezogen werden. Dabei konnte die Konzeption bei der Implementierung berücksichtigt werden und Aspekte der mehrseitigen Sicherheit und benutzergesteuerten Privatsphäre finden sich ebenfalls im Prototyp. Der Prototyp besteht aus der Sprachverarbeitungsumgebung (Cloud Services), der Mobile App und dem Privacy Provider.

Anhand der genutzte Cloud Services von Amazon konnten die Funktionalitätsanforderungen erfüllt werden. Aktuell unterstützen die sprachbasierten Cloud Services von Amazon kein Deutsch, deshalb wurde der Prototyp mit Englisch als Sprache umgesetzt. Amazon kündigte jedoch an, bald mehr Sprachen, unter anderem auch Deutsch, anzubieten. Außerdem kann durch die erfüllten DSVGO-Richtlinien der Sprachservices der Datenschutz für die Nutzer gewährleistet werden. Durch die Verwendung der sprachbasierten Cloud Services müssen sich Entwickler nicht detailliert mit der Sprachverarbeitung auseinandersetzen, sondern können auf abstrakter Ebene eine Anwendung entwickeln.  
Bevor ein Nutzer den Sprachassistenten nutzen kann, muss dieser sich authentifizieren. Somit wird der Zugriff vor nichtberechtigten Personen geschützt. In der App können Nutzer verschiedene Profile mit unterschiedlichen Daten anlegen. Hierbei ist die Verwendung von Pseudoprofilen möglich. Im Hinblick auf das Konzept zur mehrschichtigen Sicherheit ist dies wichtig, um Auswahlmöglichkeiten und Verhandlungsspielräume für den Nutzer zu schaffen. Die Nutzerdaten sind in einem Privacy Provider abgelegt. Auch für den Privacy Provider gilt das Konzept der mehrseitigen Sicherheit. Hier ist die Dezentralisierung und Verteilung von großer Bedeutung. Durch die Technologie- und Anbieterauswahl wird auf allen Ebenen der Datenschutz berücksichtigt. Die Daten sind in Deutschland, womit auch eine Rechtslage angewendet wird, die deutlich strenger bei der Datenhaltung ist. Allerdings gibt es beim Privacy Provider auch noch Optimierungspotenzial. Zum einen soll der Ressourcenzugriff konfigurierbarer durch Berechtigung, Dauer und Filterung gestaltet werden. 

In diesem Prototyp wurden nur ein paar Anwendungsfälle umgesetzt. Je nach Nutzer variieren die Anforderungen an einen Sprachassistenten. Verschiedene Anwendungen sollten anhand der Bedürfnisse eines Nutzer aktiviert oder deaktiviert werden können. Dieses Funktionalität könnte den Prototyp in der Zukunft erweitern. Dabei müssen die angebotenen Anwendungen die Nutzer über verwendeten Daten informieren und damit Transparenz  schaffen. Um ein Ökosystem zu schaffen, indem jede Anwendung auf die Daten zugreifen kann, muss ein Standard über die Datenablage geschaffen werden. Andernfalls müssen die Anwendungen die Nutzerdaten selbst verwalten und das Konzept über die Trennung von Daten und Anwendung wäre hinfällig. 

In diesem Artikel wird an verschiedenen Stellen auf das Potenzial von Sprachassistenten verwiesen. Durch die angenehme Bedienung von der Systemintelligenz bietet es einen Mehrwert im Alltag. Allerdings müssen sich verschiedene Branchen öffnen und Schnittstellen anbieten, sodass Buchungen und Reservierungen nicht nur per E-Mail oder Telefon möglich sind. Ist die Infrastruktur von Unternehmen geschaffen, werden Sprachassistenten zusätzlich an Attraktivität für die Nutzer gewinnen.