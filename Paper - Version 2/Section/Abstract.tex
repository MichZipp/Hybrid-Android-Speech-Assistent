\begin{abstract}
%\boldmath 
Aktuelle Sprachassistenten werden von großen IT-Konzernen wie Google, Amazon, Microsoft, Apple oder Baidu angeboten. Die Sprachassistenten umfassen zahlreiche Funktionalitäten, welche i. d. R. zentral in der Cloud von den Anbietern ausgeführt werden. Doch was passiert mit den Eingabedaten der Nutzer? Die Anbieter machen dazu ungenaue Angaben. Die Nutzer können sich nicht sicher sein, ob ihre Privatsphäre und Daten geschützt sind. Es stellt sich die zentrale Forschungsfrage, was mit der sprachbasierten Interaktion zwischen Nutzern und Diensten aktuell geschieht, und welche Konzepte für einen konfigurierbaren Datenschutz durch die Nutzer zukünftig vorstellbar sind. Im Artikel werden die Umfrageergebnisse vorgestellt, die von Sprachassistent-Nutzern im Rahmen dieser Forschungsarbeit ermittelt wurden. Das Ergebnis zeigt insbesondere die Zahlungsbereitschaft für einen individuell konfigurierbaren Datenschutz. Basierend auf den Erkenntnissen der ersten Veröffentlichung dieses Artikels, wird das Konzept für einen konfigurierbaren Sprachassistenten erweitert und prototypisch umgesetzt. Für den Prototyp wird eine konkrete Technologieauswahl getroffen. Anschließend wird der Artikel mit einer Bewertung des Prototyps abgerundet.
\end{abstract}