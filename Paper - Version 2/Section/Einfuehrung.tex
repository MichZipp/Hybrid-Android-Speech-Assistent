\section{Einführung}
Die Sprachsteuerung ist eine Interaktionsmöglichkeit, bei der technische Geräte durch die menschliche Sprache gesteuert werden können. Das nächste Lied, der Wecker oder auch Bestellprozesse können damit initiiert werden.  Die Experten gehen von einem wachsenden Markt für Sprachsteuerungen aus: Die Fachzeitschrift \glqq PR Newswire\grqq{} vermutet, dass Einkäufe über Sprache in den nächsten vier Jahren um das Zwanzigfache ansteigen \cite{prNewswire}. Das Magazin \glqq Campaign\grqq{} schätzt, dass in Zukunft die Suche in Browsern über die Tastatur von der Suche über Sprache ersetzt wird \cite{Campaign}. Die Sprachassistenten beinhalten solche Sprachsteuerungsservices und bilden somit die Schnittstelle zwischen Nutzern und Anwendungen. 

Die Anwendungen eines Sprachassistenten werden Apps genannt und auf einer Plattform in der Cloud ausgeführt. Die Sprachverarbeitung auf der Plattform ist anspruchsvoll, da die Spracheingabe eines Nutzers hochkomplexe Teilprozesse der Sprachverarbeitung durchläuft, bis eine passende Antwort für den Nutzer erzeugt werden kann. Aktuell werden diese Plattformen von großen Cloud-Anbietern angeboten, die über die finanziellen Mittel und das Knowhow jedes einzelnen Teilprozesses verfügen. Universitäten konzentrieren sich i. d. R. auf einen Teilprozess. Sprachassistenten werden von Amazon, Google, Microsoft oder Baidu angeboten, wobei diese viele Funktionalitäten und gute Performance bieten. Jedoch gibt es Bedenken hinsichtlich der Privatsphäre, da aus den Datenschutzerklärungen der Cloud-Anbieter nicht klar hervorgeht, was mit den Daten der Nutzer in der Cloud geschieht. Mobile Geräte wie Smartphone und Lautsprecher senden die Spracheingabe eines Nutzers für die Auswertung zum entsprechenden Cloud-Anbieter. Dabei können Daten erfasst und ggf. missbraucht werden. Die Datenverwendung wird in den Nutzungsbestimmungen angegeben, allerdings vermitteln diese eine beschränkte Aussage von den möglichen Verwendungsszenarien, wie das Profiling von Nutzern. 

Aus diesem Grund wurde eine Umfrage mit 110 Teilnehmern durchgeführt. Diese umfasst die Nutzung von Sprachassistenten, die Relevanz des Datenschutzes aus Nutzersicht und  die finanzielle Bereitschaft für mehr Datenschutz. Dabei gaben die Nutzer Anwendungen an, bei denen ihnen Datenschutz besonders wichtig ist. Die Ergebnisse werden im Kapitel \ref{sec:motivaiton} erläutert und stellen die Motivation zur Entwicklung des Konzeptes eines Sprachassistenten mit mehr Privatsphäre für den Nutzer dar. Im Kapitel \ref{sec:konzept} wird ein Konzept vorgestellt, durch das Nutzer die volle Kontrolle über ihre Daten erhalten. Die in der ersten Veröffentlichung dieses Artikels vorgestellte Architektur wurde erweitert und wird in Kapitel \ref{sec:architecure} aufgezeigt.
Anschließend wird in Kapitel \ref{sec:umsetzung} ein Prototyp vorgestellt, mit dem diese Architektur umgesetzt wurde. Der vorgestellte Prototyp ermöglicht maximale Flexibilität bei der Datenhaltung und ermöglicht dem Nutzer konfigurierbaren Datenschutz. Eine Zusammenfassung des Konzeptes, der Technologien und des entwickelten Prototyps schließen den Artikel ab. \newline